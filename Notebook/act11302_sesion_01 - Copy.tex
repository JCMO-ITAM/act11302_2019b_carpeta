%\documentclass[cjk,slidestop,compress,brown]{beamer}
\documentclass[cjk,t,compress]{beamer}
%\documentclass[cjk,slidestop,compress]{beamer}

\usepackage{amsmath}
\usepackage{amssymb}
\usepackage{amsthm}

\mode<presentation>
{
%	\usetheme{Hannover}
%	\usetheme{Goettingen}
%   \usetheme{Boadilla}
%	\usetheme{Szeged}
%	\usetheme{Singapore}
%	\usetheme{CambridgeUS}
	\setbeamercovered{dynamic}
	\usefonttheme[hoptionsi]{serif}
%  \usecolortheme[green]{structure}
}

%\usepackage[UTF-8,latin1]{inputenc}
\usepackage[spanish]{babel}
\usepackage{xcolor}
%\usepackage[dvips]{color}
%\usepackage{multicol}
%\usepackage{subfigure}

\definecolor{MyDarkBlue}{rgb}{0 0.08 0.45}
\definecolor{MyDarkOrange}{rgb}{1 0.5 0}
\definecolor{MyDarkGreen}{rgb}{0 0.25 0}
\definecolor{MyDarkTerracota}{rgb}{0.5 0 0}
\definecolor{MyDarkGrey}{rgb}{0.25 0.25 0.25}

% ---------------------------------------------
%				Fields
% ---------------------------------------------
\newcommand{\Indicator}{\operatorname{\mathds{1}}}
\newcommand{\Indic}{\operatorname{\field{I}}}
\newcommand{\field}[1]{\mathbb{#1}} 
\newcommand{\C}{\field{C}}
\newcommand{\E}{\field{E}} 
\newcommand{\R}{\field{R}}
\newcommand{\inccoin}{\Delta c}
\newcommand{\incy}{\Delta y}

% ---------------------------------------------
%				Observables
% ---------------------------------------------
\newcommand{\boldx}{\boldsymbol{x}}
\newcommand{\boldX}{\boldsymbol{X}}
\newcommand{\boldy}{\boldsymbol{y}} 
\newcommand{\boldY}{\boldsymbol{Y}}
\newcommand{\boldu}{\boldsymbol{u}} 
\newcommand{\boldU}{\boldsymbol{U}}
\newcommand{\boldzero}{\boldsymbol{0}} 
\newcommand{\boldA}{\boldsymbol{A}} 
\newcommand{\boldW}{\boldsymbol{W}}

% ---------------------------------------------
%				Parameters
% ---------------------------------------------
\newcommand{\piexp}{\pi^{e}} 
\newcommand{\pidelta}{\delta^{\pi}} 

\newcommand{\rhoexp}{\rho^{e}} 
\newcommand{\rhodelta}{\delta^{\rho}} 

\newcommand{\tauexp}{\tau^{e}} 
\newcommand{\taudelta}{\delta^{\tau}} 

\newcommand{\thetaexp}{\theta^{e}} 
\newcommand{\thetadelta}{\delta^{\theta}} 

\newcommand{\tk}{t:(t+k)} 

\newcommand{\SigmaB}{\boldsymbol{\Sigma}} 
\newcommand{\sigmaB}{\boldsymbol{\sigma}} 
\newcommand{\PhiB}{\boldsymbol{\Phi}} 
\newcommand{\phiB}{\boldsymbol{\phi}} 
\newcommand{\bB}{\boldsymbol{b}} 
\newcommand{\boldsigma}{\boldsymbol{\sigma}} 
\newcommand{\boldSigma}{\boldsymbol{\Sigma}} 
\newcommand{\boldphi}{\boldsymbol{\phi}} 
\newcommand{\boldPhi}{\boldsymbol{\Phi}} 
\newcommand{\boldbeta}{\boldsymbol{\beta}}
\newcommand{\boldBeta}{\boldsymbol{\Beta}}
\newcommand{\boldtheta}{\boldsymbol{\theta}}
\newcommand{\boldTheta}{\boldsymbol{\Theta}}

%   --------------------------------
%               Notation
%   --------------------------------
\newcommand{\Lik}{\mathrm{lik}}
\newcommand{\diagop}{\text{diag}} 
\newcommand{\logistic}{\operatorname{\text{logistic}}}
\newcommand{\logit}{\operatorname{\text{logit}}}
\newcommand{\dom}{\operatorname{\text{dom}}}
\newcommand{\ran}{\operatorname{\text{ran}}}
\renewcommand{\inf}{\operatorname{\text{inf}}}
\newcommand{\ind}{\operatorname{\text{ind}}}
\newcommand{\iid}{\operatorname{\text{iid}}}
\newcommand{\cind}{\operatorname{\text{c.i.}}}
\newcommand{\wport}{\text{w}}

%\newcommand{\Pr}{\field{P}}
\newcommand{\dd}{\mathrm{d}}
\newcommand{\Borel}{\operatorname{\mathscr{B}}}
\newcommand{\Filtration}{\operatorname{\mathscr{F}}}
\newcommand{\Expec}{\operatorname{\field{E}}}
\newcommand{\Var}{\operatorname{\text{Var}}}
\newcommand{\Prec}{\operatorname{\text{Prec}}}
\newcommand{\Cov}{\operatorname{\text{Cov}}}
\newcommand{\Corr}{\operatorname{\text{Corr}}}
\newcommand{\varmodel}{\operatorname{\text{VAR}}}

\renewcommand{\sin}{\operatorname{\text{sin}}}
\renewcommand{\cos}{\operatorname{\text{cos}}}
\renewcommand{\tan}{\operatorname{\text{tan}}}
\renewcommand{\arctan}{\operatorname{\text{arctan}}}
\renewcommand{\exp}{\operatorname{\text{exp}}}
\renewcommand{\log}{\operatorname{\text{log}}}
\renewcommand{\arg}{\operatorname{\text{arg}}}
\renewcommand{\min}{\operatorname{\text{min}}}
\renewcommand{\max}{\operatorname{\text{max}}}
\renewcommand{\lim}{\operatorname{\text{lim}}}
\newcommand{\limite}{\operatornamewithlimits{lim}}
\renewcommand{\det}{\operatorname{\text{det}}}
\renewcommand{\dim}{\operatorname{\text{dim}}}
\newcommand{\sign}{\operatorname{\text{sign}}}
\newcommand{\argmax}{\operatornamewithlimits{arg\,max}}
\newcommand{\argmin}{\operatornamewithlimits{arg\,min}}
\newcommand{\diagbloq}{\operatorname{\text{diagbloq}}}
\newcommand{\rr}{\operatorname{\textsc{R}}}

\newcommand{\ie}{{\it i.e. }} 
\newcommand{\aka}{{\it a.k.a. }} 
\newcommand{\eg}{{\it e.g. }} 
\newcommand{\ala}{{\it \`{a}~la }} 
\newcommand{\visavis}{{\it vis-\`{a}-vis }}

\newcommand{\inpc}{\text{INPC}}
\newcommand{\igae}{\text{IGAE}}
\newcommand{\pib}{\text{PIB}}
\newcommand{\tiie}{\text{TIIE}}
\newcommand{\export}{\text{Exportaciones}}
\newcommand{\import}{\text{Importaciones}}
\newcommand{\usip}{\text{IP}^{\usa}}
\newcommand{\remMX}{\text{Rem}^{\usa}}
\newcommand{\itaen}{\text{ITAE}}
\newcommand{\itaee}{\text{ITAEE}}
\newcommand{\usa}{\text{EEUU}}
\newcommand{\dsge}{\operatornamewithlimits{DSGE}}

% ---------------------------------------------
%				Distributions
% ---------------------------------------------

\newcommand{\WiD}{\operatorname{\text{Wi}}}
\newcommand{\WeD}{\operatorname{\text{We}}}
\newcommand{\WeNormD}{\operatorname{\text{We-N}}}
\newcommand{\ExpD}{\operatorname{\text{Exp}}}
\newcommand{\GeoD}{\operatorname{\text{Geo}}}
\newcommand{\StD}{\operatorname{\text{St}}}
\newcommand{\NormD}{\operatorname{\text{N}}}
\newcommand{\GaD}{\operatorname{\text{Ga}}}
\newcommand{\BeD}{\operatorname{\text{Be}}}
\newcommand{\UniD}{\operatorname{\text{U}}}
\newcommand{\DirD}{\operatorname{\text{Dir}}}
\newcommand{\IG}{\operatorname{\text{InG}}}
\newcommand{\IncGa}{\operatorname{\text{IGa}}}
\newcommand{\IGa}{\operatorname{\text{InGa}}}
\newcommand{\PoD}{\operatorname{\text{Po}}}
\newcommand{\BS}{\operatorname{\text{BS}}}
\newcommand{\DP}{\operatorname{\text{DP}}}


%   --------------------------------
%               Directories
%   --------------------------------
\graphicspath{{../figures/}}
\DeclareGraphicsExtensions{.pdf,.jpeg,.png,.jpg}

\title[C\'alculo Actuarial III]
{	ACT-11302: C\'alculo Actuarial III\\
	{\large Sesi\'on 01 - Modelos de Probabilidad}
}

\author[Mart\'inez-Ovando]{
{	\footnotesize
	\textcolor{MyDarkGreen}{Juan~Carlos Mart\'inez-Ovando}}
}

\institute[ITAM]
{	\textcolor{MyDarkGrey}{
	ITAM}
}

\date[ ] % (optional)
{	\scriptsize
	\textcolor{MyDarkGrey}{Oto\~no 2019}
}

%	--------------------------------------------
%									Titlepage
%	--------------------------------------------
\begin{document}
\sffamily
\frame
	{
   \frametitle{}
	\titlepage
	}
	
%	--------------------------------------------
%									Contents
%	--------------------------------------------
\frame{
	\frametitle{Agenda}
	\scriptsize
	\tableofcontents[section]
	}


%	--------------------------------------------
%	section
%	--------------------------------------------
\section{Modelo de probabilidad}
	%	-------------		FRAME		--------------------
	\frame
	{
   \frametitle{}
		
		\vspace{5.5cm}
		\begin{flushright}
			\textcolor{MyDarkBlue}{\Large \bf Modelo de probabilidad}
		\end{flushright}
	}
	
	%	--------------------------------------------
	%	section
	%	--------------------------------------------
	\subsection{Definici\'on}
	%	-------------		FRAME		--------------------
	\frame
	{
    \frametitle{Definici\'on}
  		{\scriptsize  	
		
		\begin{block}<+->{Notaci\'on}
	  	{
	  	\begin{itemize}
	  		\item $X$ se refiere a una variable aleatoria observable (discreta o continua) 
	  		\item $x$ se refiere a un valor espec\'ifico de esta variable
	  	\end{itemize}
	  	}
 		\end{block}  		

	  	\vspace{0.3cm}
		\begin{block}<+->{Modelo de probabilidad}
	  	{
		Sin p\'erdida de generalidad, refir\'amonos a $X$. El modelo de probabilidad se define como la distribuci\'on de probabilidades de $X$ indizada por $\theta$, i.e.
		\begin{equation}
			X \sim F(x|\theta).
		\end{equation}
	  	\vspace{0.1cm}
	  	El $soporte$ de $X$, denotado por $\mathcal{X}$, se define como,	  	
		\begin{equation}
			\mathcal{X}=\{x:F(x|\theta)>0\},
		\end{equation}
		donde $\mathcal{X}$ forma un subconjunto de un espacion Euclidiano de dimensi\'on finita.
	  	\vspace{0.1cm}
	  	El par\'ametro $\theta$, toma valores en el espacio parametral $\Theta$ (generalmente de dimensi\'on finita).
		}
 		\end{block}  		

		}
	}

	%	-------------		FRAME		--------------------
	\frame
	{
    \frametitle{Definici\'on}
  		{\scriptsize  	
		
		\begin{block}<+->{Densidades y masa de probabilidad}
	  	{
		\begin{enumerate}
			\item Cuando $X$ es absolutamente continua, $F(x|\theta)$ admite una densidad, $f(x|\theta)$, tal que 
			$$
			F(x|\theta)=\int_{-\infty}^{x}f(s|\theta)ds,
			$$
			implicando que el soporte no tenga \'atomos, i.e.
			$$
			\Pr(X=x)=0$$
			para todo $x \in \mathcal{X}$.
			
			\item Cuando $X$ es del tipo discreto, el soporte $\mathcal{X}$ est\'a formado solamente por \'atomos, i.e. valores especificos de $X$, digamos $\mathcal{X}=\{x^{*}_1,\ldots,x^{*}_n\}$ tales que 
			$$
			\Pr(X=x^{*}_i)=p_i>0,
			$$
			para todo $x^{*}_i \in \mathcal{X}$, con
			$$
			\sum_{i=1}^{n}p_i=1.
			$$ 
		\end{enumerate}
		}
 		\end{block}  		

		}
	}


	%	-------------		FRAME		--------------------
	\frame
	{
    \frametitle{Definici\'on}
  		{\scriptsize  	
		
		\begin{block}<+->{Densidades y masa de probabilidad}
	  	{
		\begin{enumerate}
			\item[3] Cuando $X$ es del tipo mixta, el modelo de probabilidad admite una parte absolutamente continia al mismo tiempo de admitir una parte discreta, i.e.
			\begin{equation}
				\Pr(X\leq x)=F(x|\theta) = F_c(x|\theta_c)+ \sum_{x^{*}_k \leq x} p(X=x^{*}_k|\theta_d),
			\end{equation}
			donde
			\begin{itemize}
				\item $F_c(\cdot)$ es el componente continuo de la distribuci\'on
				\item $\{x^{*}_k\}_{k\geq 1}$ son los \'atomos de la distribuci\'on
				\item $\theta_c$ y $\theta_d$ son los par\'ametros asociados con la parte continua y discreta, respectivamente.
			\end{itemize}
		\end{enumerate}
		}
 		\end{block}  		

		\begin{block}<+->{Densidades y masa de probabilidad}
	  	{
	  	En este tipo de distribuciones, el soporte $\mathcal{X}$ est\'a formado de una parte absolutamente continua (sin \'atomos), $\mathcal{X}_{c}$ , y una parte discreta (formada solo de \'atomos), $\mathcal{X}_{d}$, i.e.
		$$
		\mathcal{X}=\mathcal{X}_{c}\cup\mathcal{X}_d.
	  	$$
		}
 		\end{block}  		

		}
	}

	%	-------------		FRAME		--------------------
	\frame
	{
    \frametitle{Ejemplo}
  		{\scriptsize  	
		
		\begin{block}<+->{Ejemplo}
	  	{
		Pensemos en el modelo de probabilidad con un \'atomo en $\{0\}$ que admite la posibilidad de tomar valores en la recta real positiva, i.e.
		$$
		\mathcal{X}=\{0\}\cup(0,\infty).
		$$
		}
 		\end{block}  		

		\begin{block}<+->{Modelo de probabilidad}
	  	{
		El modelo de probabilidad estar\'a definido por una masa de probabilididad en $\{0\}$, i.e.
		$$
		\Pr(X=0)=\Pr(X\in\{0\})=\theta_0,
		$$
		y una densidad para la parte continua,
		$$
		f(x|\theta_c)=\theta_c\exp\{-x\theta_c\}\mathbb{I}_{(0,\infty)}(x),
		$$
		con $0<\theta_0<1$ y $\theta_c>0$.
		}
 		\end{block}  		

		\begin{block}<+->{Ejercicio}
	  	{
	  	\textcolor{red}{?`Qu\'e forma toma $F(x|\theta)$ y qui\'en es $\theta$?}
		}
 		\end{block}  		

		}
	}


	%	--------------------------------------------
	%	section
	%	--------------------------------------------
	\subsection{Verosimilitud}
	%	-------------		FRAME		--------------------
	\frame
	{
    \frametitle{Ejemplo}
  		{\scriptsize  	
		
Ahora, incorporaci\'on de datos (informaci\'on)...		

\vfill

	  	Consideremos un conjunto de datos $X_1=x_1,\ldots,X_n=x_n$ (no \'atomos) en el caso absolutamente continuo.
	  	

		}
	}
	
	%	-------------		FRAME		--------------------
	\frame
	{
    \frametitle{Verosimilitud}
  		{\scriptsize  	
		
		\begin{block}<+->{Funci\'on de verosimilitud}
	  	{
	  	\begin{itemize}
	  		\item Enfoque frecuentista: Independencia
	  		\begin{equation}
	  		 \Pr(X_1=x_1,\ldots,X_n=x_n;\theta) = \prod_{i=1}^{n} f(x_i;\theta).
	  		\end{equation}
	  		\item Enfoque bayesiano: Independencia condicional 
	  		\begin{equation}
	  		 \Pr(X_1=x_1,\ldots,X_n=x_n) = \int \prod_{i=1}^{n} f(x_i|\theta) \pi(\theta) \dd \theta.
	  		\end{equation}
	  	\end{itemize}
	  	}
 		\end{block}  		

	  	\vspace{0.3cm}
		\begin{block}<+->{Ejercicio}
	  	{
	  	\textcolor{red}{?`C\'omo ser\'a la expresi\'on de la funci\'on de verosimilitud en el caso discreto y tipo mixta?}
	  	}
 		\end{block}  		
		}
	}

	%	-------------		FRAME		--------------------
	\frame
	{
    \frametitle{Verosimilitud}
  		{\scriptsize  	
		
		\begin{block}<+->{Tipos de datos}
	  	{
	  	\begin{itemize}
	  		\item Las expresiones anteriores son correctas cuando los datos son exactos. 
	  		\item Cuando trabajamos con datos agrupados en $\Re_+$, modificamos el soporte $\mathcal{X}$ por una partici\'on $\{c_j\}_{j=1}^{J}$ tal que, 
	  		\begin{equation}
	  			c_1 < c_2 < \ldots <c_J,
	  		\end{equation}
	  		sustituyendo $\mathcal{X}$ por el conjunto,
	  		\begin{equation}
	  			\mathcal{C}=\{ (c_{j},c_{j+1}]: c_j < c_{j+1}, j=1,\ldots,J\}.
	  		\end{equation}
	  		
	  	\end{itemize}
	  	}
 		\end{block}  		

	  	\vspace{0.3cm}
		\begin{block}<+->{Ejercicio}
	  	{
	  	\textcolor{red}{?`C\'omo ser\'a la expresi\'on de la funci\'on de verosimilitud para datos agrupados?}
	  	}
 		\end{block}  		

		}
	}

	%	--------------------------------------------
	%	section
	%	--------------------------------------------
	\subsection{Conjugacidad}
	%	-------------		FRAME		--------------------
	\frame
	{
    \frametitle{Distribuciones conjugadas}
  		{\scriptsize  	
		
	  	En el an\'alisis bayesiano de datos, el uso de familias conjugadas entre $f(x|\theta)$ y $\pi(\theta)$ es de utilidad para obtener expresiones ana\l\'iticas cerradas en el proceso de aprendizaje.
	  	
		\begin{block}<+->{Familia Exponencial}
	  	{
	  	Las familias conjugadas est\'an definidas dentro de la  Familia Exponencial de Distribuciones (lineal), para las que la funci\'on de densidad o masa de probabilidad admiten la siguiente expresi\'on,
	  	\begin{equation}
	  		f(x|\theta) = p(x) q(\theta)^{-1} exp\{-\theta x\},
	  	\end{equation}
	  	considerando que el soporte $\mathcal{X}$ no depende de $\theta$.
	  	}
 		\end{block}  		

	  	\vspace{0.3cm}
		\begin{block}<+->{Prior conjugada}
	  	{
	  	Las distribuci\'on inicial conjugada para la representaci\'on atenrior toma la forma,
	  	\begin{equation}
	  		\pi(\theta) = c(k_0,m_0)q(\theta)^{-k_0} exp\{-\theta m_0\},
	  	\end{equation}
	  	donde $k_0$ y $m_0$ son hiper par\'ametros.
	  	}
 		\end{block}  		

		}
	}

	%	--------------------------------------------
	%	section
	%	--------------------------------------------
	\subsection{Predicci\'on}
	%	-------------		FRAME		--------------------
	\frame
	{
    \frametitle{Predicci\'on}
  		{\scriptsize  	
		
		\begin{block}<+->{Enfoque frecuentista}
	  	{
	  	Bajo el enfoque frecuentista, la predicci\'on de un valor futuro de $X$, $X^f$, se obtiene a trav\'es de la imputaci\'on del EMV de $\theta$ en el modelo, i.e.
	  	\begin{equation}
	  		X^f|x_1\ldots,x_n \sim f(x^f|\widehat{\theta}_n),
	  	\end{equation}
	  	donde $\widehat{\theta}_n=\widehat{\theta}_n(x_1\ldots,x_n)$.
	  	}
 		\end{block}  		

	  	\vspace{0.3cm}
		\begin{block}<+->{Enfoque bayesiano}
	  	{
	  	Bajo el enfoque bayesiano, la predicci\'on se obtiene usando argumentos probabilistas, como
	  	\begin{equation}
	  		p(x^f|x_1\ldots,x_n) = \int_{\Theta} f(x^f|\theta) \pi(\theta|x_1\ldots,x_n)\dd \theta,
	  	\end{equation}
	  	donde $\pi(\theta|x_1\ldots,x_n) \propto f(x_1\ldots,x_n|\theta)\pi(\theta)$ es la distribuci\'on de $\theta$ actualizada con la informaci\'on contenida en $x_1\ldots,x_n$.
	  	}
 		\end{block}  		

	  	\vspace{0.3cm}
		\begin{block}<+->{Ejercicio}
	  	{
	  	\textcolor{red}{Muestra que el modelo Bernoulli-beta, visto en las clases previas, es un tipo de distribuciones conjugadas.}
	  	}
 		\end{block}  		

		}
	}

	%	--------------------------------------------
	%	section
	%	--------------------------------------------
	\subsection{Intercambiabilidad}
	%	-------------		FRAME		--------------------
	\frame
	{
    \frametitle{Intercambiabilidad}
  		{\scriptsize  	
		
		\begin{block}<+->{Definici\'on}
	  	{
	  	Se dice que un conjunto (numerable) de variables  aleatorias $\{X_j\}_{j=1}^{\infty}$ es intercambiabiable con respecto a $\Pr$ si para todo $n$ finito,
	  	\begin{equation}
	  		\Pr(X_1,\ldots,X_n)=\Pr(X_{\sigma(1)},\ldots,X_{\sigma(n)}),
	  	\end{equation}
	  	donde $(\sigma(1),\ldots,\sigma(n))$ es cualquier permutaci\'on del vector $(1,\ldots,n)$.
	  	}
 		\end{block}  		

	  	\vspace{0.3cm}
		\begin{block}<+->{Comentarios}
	  	{
	  	\begin{itemize}
	  		\item Cualquier sucesi\'on de variables aleatorias $\iid$ es naturalmente intercambiable.
	  		\item La noci\'on de intercambiabilidad, como la de independencia, se refiere a que el orden de la informaci\'on es irrelevante (i.e. los resultados anal\'iticos son
invariantes ante permutaciones).
	  	\end{itemize}
	  	}
 		\end{block}  		

	  	\vspace{0.3cm}
		\begin{block}<+->{Ejercicio}
	  	{
	  	\textcolor{red}{Describe un ejemplo donde los datos podr\'ian asociarse con el supuesto de
intercambiabilidad, mas no con el de independencia. Describe tambi\'en un ejemplo donde ni 
intercambiabilidad ni independencia ser\'ian supuestos viables.}
	  	}
 		\end{block}  		
		}
	}

	%	-------------		FRAME		--------------------
	\frame
	{
    \frametitle{Intercambiabilidad}
  		{\scriptsize  	
		
		\begin{block}<+->{Representaci\'on: de~Finetti}
	  	{
	  	Una consecuencia del supuesto de intercambiabilidad (numerable) es el teorema de representaci\'on en el que se admite que para toda sucesi\'on de variables aleatorias
intercambiables, para toda $n$ finita, se tiene que existe  un ente estoc\'astico $\theta \in \Theta$
acompa\~nado de una medida de probabilidad $\Pi$, tal que
	  	\begin{equation}
	  		\Pr(X_1,\ldots,X_n)=\int_{\Theta} \left\{ \prod_{j=1}^{n} \Pr(X_j|\theta) \right\} \Pi(\dd \theta),
	  	\end{equation}
	  	donde $(\sigma(1),\ldots,\sigma(n))$ es cualquier permutaci\'on del vector $(1,\ldots,n)$.
	  	}
 		\end{block}  		

	  	\vspace{0.3cm}
		\begin{block}<+->{Comentarios}
	  	{
	  	\begin{itemize}
	  		\item El resultado anterior es de {\em existencia}, i.e. no nombra c\'omo se lleva a cabo tal representaci\'on.
	  		\item Para un conjunto de variables aleatorias intercambiables existen m\'as de una posible representaci\'on como la anterior (en t\'erminos de diferentes
especificaciones de $\theta$ y/o de $\Pi$).
			\item Este teorema de representaci\'on brinda una interpretaci\'on al paradigma bayesiano de inferencia.
	  	\end{itemize}
	  	}
 		\end{block}  		

		}
	}

	%	--------------------------------------------
	%	section
	%	--------------------------------------------
	\subsection{Ejercicio para casa}
	%	-------------		FRAME		--------------------
	\frame
	{
    \frametitle{Ejercicio para casa}
  		{\scriptsize  	
		Consideremos el caso sencillo donde $X$ es discreta con soporte en $\mathcal{X}=\{0,1\}$.
		\begin{enumerate}
		  \item Defina el modelo de probabilidad para $X$. 
		  \item Identifique el par\'ametro del modelo.
		  \item Defina la varosimilitud para el par\'ametro basado en el supuesto de $independencia$ con base en tres datos $x_1=1$, $x_2=1$ y $x_3=1$.
		  \item Calcule la distribuci\'on predictiva para $X_4$.
		  \item Examine la forma gen\'erica de la funci\'on de verosimilitud para el par\'ametro del modelo.
		  \item Identifique la distribucion $\Pi$ conjugada.
		  \item Calcule la distribuci\'on predictiva para $X_4$ usando los mismos datos empleados anteriormente, $x_1,x_2,x_3$.
		\end{enumerate}
		}
	}

%	--------------------------------------------
%	--------------------------------------------
%	section
%	--------------------------------------------
%	--------------------------------------------
\section*{ }
\frame
{\scriptsize
	\begin{center}
		\textcolor{MyDarkOrange}{\huge }
		
		\vspace{1.7cm}
		\textcolor{MyDarkBlue}{\Large Gracias por su atenci\'on...}
		
		\vspace{1.3cm}
		\textcolor{MyDarkGreen}{\large \tt juan.martinez.ovando@itam.mx}
		
	\end{center}
}
\end{document}

%
%	--	FIN	--